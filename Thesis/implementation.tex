\chapter{Implementation}


\section{WebExtensions}
\paragraph{Write a blah blah paragraph, can be short, about the basic structure of WebExtension. The following are Manifest Keys that will be required for my implementation.}

\begin{enumerate}
\item \emph{manifest\_version}: Always 2
\item \emph{name}: Name of the implementation
\item \emph{description}: Description of the extension
\item \emph{version}: The version of the software
\item \emph{author}: The name of the developer
\item \emph{applications}: Also known as the \emph{brower\_specific\_settings} contains keys that are specific to the particular host application. 
\begin{center}
\begin{minted}{javascript}
"background": {
    "page": "background.html"
},
\end{minted}
\end{center}
\end{enumerate}


\paragraph{In Thunderbird, all WebExtension API can be accessed through the browser.* namespace, as with Firefox, but also through the messenger.* namespace, which is a better fit for Thunderbird.}\footnote{From website: https://developer.thunderbird.net/add-ons/mailextensions/hello-world-add-on/using-webextension-apis}

\paragraph{messenger.tabs.query}
\subparagraph{The tabs API provides access to Thunderbird's tabs. In some cases, since the call is executed from where it is called from, it may be necessary to execute tabs \emph{message\_display\_action} and not from inside the tab we are looking for.}

\paragraph{messenger.messageDisplay.getDisplayedMessage()}
\subparagraph{the \emph{getDisplayedMessage} method of the \emph{messageDisplay API} provides access to the currently viewed message in a given tab. It returns a promise for a \emph{MessageHeader} object from the \emph{message API} with basic imformation about the message.}\footnote{The getDisplayMessage method requires the messageRead permission, which needs to be added ot the permissions key of the manifest.json file.}

\paragraph{messenger.messages.getFull()}
\subparagraph{The \emph{getFull()} method returns a Promise for a MessagePart object, which relates to messages containing multiple MIME parts. The headers member of the part returned by getFull includes the headers of the message.}

\paragraph{Steps to create an addon in Thunderbird.}

\subsection{Using a background page}
% Adding the following, may or may not use, but learning them can't hurt?
\paragraph{First, we need to add "backgrounds" to our manifest.}

\begin{center}
\begin{minted}{javascript}
"background": {
    "page": "background.html"
},
\end{minted}
\end{center}

\paragraph{Additionally, we'll need an html and javascript document, as well as keep the manifest updated.}

\begin{center}
\begin{minted}{html}
<!DOCTYPE html>
<html>
<head>
    <meta charset="utf-8"/>
    <script src="background.js" type="module"></script>
</head>
</html>
\end{minted}
\end{center}


\begin{enumerate}
\item Create manifest.json
\item Have it reference a folder (that it is outside of), like "MyAddon", which contains
\begin{itemize}
\item myaddon.html
\item myaddon.css
\item myaddon.js
\end{itemize}
\item Then, create another folder named "images", for images.
\item To install, open thunderbird
\item click on the hamburger button on the right side of tb
\item select "Debug Add-ons"
\item blick on "Load Temporary Add-on..." button
\item navigate to the previously created manifest.json file.
\end{enumerate}

\paragraph{The Add-on should be fully functional now.}


\section{Crypto JS}