% This if from notes in talking with Dr. Weber-Wulff on May 9th.
% Audience is someone holding a bachelor's in Computer Science
% I don't need to tell them what HTML is, or a programming language
% I do however need to give a quick view of various other aspect of
% the report that may be more specific, in my case the Thunderbird Webextensions API
% encryption, etc..

% Do I need to mention motivation? Email and Email client can go under that.
% what is another word for motivation

% Also, according to Dr. Weber-Wulff, the rule of them for quoting works is thus
% if it's something EVERYONE knows, then I don't need to note it.

% On the other hand, if I do need to quote it think of this:

% { a Beginning
% } an End
% => and a Source

\chapter{Introduction}
%1. Establish your territory - begin to elaborate what your topic is about
\paragraph{The digital age has fully taken hold into our societies. We do everything in some for or another of digital media: create art, science, communicate, create and share memories, play games, write thesis reports. There is basically no limit to what people do with their computers.}
\paragraph{Related to this, the growth of the internet has more and more pushed our activities online. In it's nascent, this was not thought to be a major as things like Chinese or Russian Hackers, or even U.S. government intrusions were not really thought to be more than fringe blogger material. Especially, and likely most damaging, was the basic expectation of private communication. Edward Snowden's revelations about the "Five Eyes" intelligence alliance, and cooperation in the collection of all online communication, social media, phone data, etc. No online communication has been considered safe ever since.}


%2. Establish a niche - why did i pick my topic in the first place
\paragraph{Mozilla has tried to support end-to-end encryption (E2EE? for a long time, it has been faced with a major obstacles:}

\begin{itemize}
\item Setting the PGP add-on Enigmail was too technical
\item Generating keys was too technical
\item Even if conditions 1. \& 2. were fulfilled, it was especially uncommon that anyone else you would want to converse with would have gone through the trouble to setup a client or keys for themselves
\item Mozilla is in the process of using OpenPGP build-in to the client, but that also has problems, most obviously, you again need new keys (granted easier to setup this time)
\item and, again, both people must have generated keys (again
\end{itemize}



%3. Introduce my current research
\paragraph{This project is centered on the implementation of an Email Add-on that will allow end-to-end encrypted (E2EE) communication. More specifically, it will focus on the Mozilla Thunderbird client, for the simple fact that I have personally used it for over ten years, it's free, open-source, and cross platform. While I grant that not everyone uses Thunderbird, at least there should be no shortage of users, and theoretically anyone can get it easily, for free.}

\paragraph{Ultimately, this project aims to offer a simple, albeit \emph{not} perfect solution for those interested in privacy, that don't have the technical expertise to engage in key creation, exchanges or have zero knowledge about encryption. The \author{Esteban Licea} will demonstrate the advantages and disadvantages of various implementations strategies, and implement a solution that offers, hopefully, a viable encryption option that will fulfill some use cases.}


