% This if from notes in talking with Dr. Weber-Wulff on May 9th.
% Audience is someone holding a bachelor's in Computer Science
% I don't need to tell them what HTML is, or a programming language
% I do however need to give a quick view of various other aspect of
% the report that may be more specific, in my case the Thunderbird Webextensions API
% encryption, etc..

% Do I need to mention motivation? Email and Email client can go under that.
% what is another word for motivation

% Also, according to Dr. Weber-Wulff, the rule of them for quoting works is thus
% if it's something EVERYONE knows, then I don't need to note it.

% On the other hand, if I do need to quote it think of this:

% { a Beginning
% } an End
% => and a Source

\chapter{Introduction}

\paragraph{The digital age has fully absorbed our societies. We do everything in some form or another on digital media: create art, science, communicate, create and share memories, play games, and write thesis reports with our computers. There is basically no limit to what people do with their computers.}

\paragraph{Proportional to this growth, the internet's influence on our lives has also ballooned. Our activities have been pushed more and more online, and onto "the cloud." Originally, few bothered to think about privacy. Most damaging, perhaps, was the erroneous expectation of private communication. Edward Snowden's revelations about the "Five Eyes" intelligence alliance, between the United States, the United Kingdom, Canada, Australia, and New Zealand and their the collection of all online communication, social media, and phone data removed any doubt about expectations to individual privacy. No online communication, or online activities in general, has been considered safe ever since.}


\section{Background}

\paragraph{In the realm of email communication, Pretty Good Privacy (PGP) has existed for decades. It is predicated on the exchange of user created public keys (in combination with private, non-exchanged keys). However, it has a few inherent obstacles. First, there is a technical requirement to create and exchange keys. In order to facilitate this, additional client software must be installed. Additionally, several challenging steps beyond the scope of the average end user will need to be completed, like selecting encryption algorithm, size of key, etc. Originally, Thunderbird relied on an add-on, Enigmail, to create, manage, and exchange keys. The author used this add-on for many years. And, while it was satisfactory, it was plain to see that it was not without it's technical requirements.}

\paragraph{Starting with Thunderbird 78 (2020), Mozilla implemented OpenPGP as part of its core client software, and dropped support for all add-ons not using MailExtensions (which includes Enigmail). There were a few reasons for this, including Mozilla's desire to simplify the process. But, also the desire to move away from the PGP trademarked software. Nevertheless, the OpenPGP feature is disabled by default, and is still considered a work in progress. All other encryption add-ons found on Thunderbird's extensions page or searching through Github were considered to be in a testing or experimental phase.}

\section{Problem}

\paragraph{Mozilla has tried to support end-to-end encryption (E2EE) for a long time, but it has been faced with major obstacles, partly mentioned above, including:}

\begin{itemize}
\item Setting up the Enigmail add-on was too technical
\item Generating keys was too technical
\item Even if conditions 1. \& 2. were fulfilled, its uncommon that others have created their own public keys
\item Mozilla is in the process of using OpenPGP, a built-in component to the Thunderbird client, but that also has problems: most obviously, you need new keys (granted, easier to set up this time)
\item and, again, both people must have generated keys (again, easier this time)
\item Lastly, this OpenPGP built-in component, is still in its early developmental stages.
\end{itemize}

\paragraph{Thus, the problem: How can Alice send an encrypted email to someone that does not have any type of public key available?}

\section{Solution}
%3. Introduce my current research

\paragraph{The bachelor thesis candidate intends to research and develop a Thunderbird Add-on, that will allow Alice to send an encrypted message to Bob. Bob is not a tech savvy person, and is clueless about encryption technology. So, the idea of learning, installing, and setting up any types of keys, for him, is overwhelming. However, they do communicate regularly, so Alice can just whisper a one time password to him. Subsequently, Alice could then use the developed add-on to encrypt an email message for Bob. Which, he then could decrypt using the same add-on, and the previously agreed upon password.}\footnote{Alice and Bob are fictional characters commonly used as placeholders in discussions about cryptographic systems and protocols.}

\paragraph{This thesis will focus on the Mozilla Thunderbird client, for the simple fact that it's free, open-source, and cross platform. While I grant that not everyone uses Thunderbird, at least there should be no shortage of users, and theoretically anyone can get it easily, for free.}

\paragraph{Ultimately, this project aims to offer a simple, albeit \emph{not} perfect solution for those interested in privacy, that don't have the technical expertise to engage in key creation, exchanges or have zero knowledge about encryption. The author will demonstrate the advantages and disadvantages of various implementations strategies, and implement a solution that offers, hopefully, a viable encryption option that will fulfill some use cases.}

\paragraph{Research will dictate the best implementation strategy.}

\section{Methods applied}
%1. Establish your territory - begin to elaborate what your topic is about

%2. Establish a niche - why did i pick my topic in the first place



\paragraph{The methods and tools used to solve this research inquiry will include:}

\begin{enumerate}
\item Research literature and books
\item Online learning resources
\item Mozilla's WebExtensions API and JS Encryption APIs
\item Guidance from Mentors, and fellow Thunderbird add-on developers
\item Visual Studio Code coding
\item Github for version control
\item Latex for composing text
\item Jira for project management
\end{enumerate}

\paragraph{After the research has been completed, all coding will proceed using a test driven development approach. Thunderbird add-ons are based on MailExtension technology, which are created using the follow standard languages:}

\begin{enumerate}
\item HTML
\item CSS
\item Javascript
\end{enumerate}







